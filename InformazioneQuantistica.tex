\documentclass[12pt]{report} %Modalità draft settata qui


%%%HEADER%%%
\usepackage{subfiles} %Struttura modulare
\newcommand{\onlyinsubfile}[1]{#1}
\newcommand{\notinsubfile}[1]{}

\usepackage[ddmmyyyy,hhmmss]{datetime} %Data di compilazione

%%PACKAGES
\usepackage[usenames, dvipsnames, table]{xcolor} %colori
\usepackage[utf8]{inputenc}
\usepackage[T1]{fontenc}
\usepackage{lmodern}

\usepackage{amsmath}
\usepackage{amsthm}
\usepackage{amstext}
\usepackage{amsbsy}
\usepackage{amsfonts}
\usepackage{comment}
\usepackage{wrapfig}
\usepackage{booktabs}
\usepackage{braket}
\usepackage{pgf,tikz}
\usepackage{mathrsfs}
\usetikzlibrary{arrows}
%\usepackage{subfigure}
\usepackage{xspace}
\usepackage{gnuplottex}
\usepackage{epstopdf}
\usepackage{marginnote}
\usepackage{float}
\usetikzlibrary{tikzmark}
\usepackage{graphicx}
\usepackage{cancel}
\usepackage{bm}
\usepackage{mathtools}
\usepackage{ragged2e}
\usepackage[stable]{footmisc}

\usepackage[symbol=$\wedge$,numberlinked=false]{footnotebackref}
\usepackage{enumerate}
\usepackage{mathdots}
\usepackage[framemethod=tikz]{mdframed}
\PassOptionsToPackage{table}{xcolor}
\usepackage{soul}
\usepackage{enumerate}
\usepackage{mathdots}
\usepackage[framemethod=tikz]{mdframed} %Added 16/10
\usepackage[italian]{babel} %Added 16/10
\usepackage{amssymb} %Added
\usepackage{enumitem}
\usepackage{array}
\usepackage{siunitx}
\usepackage{mathdots}
\usepackage{yhmath}
\usepackage{color}
\usepackage{multirow}
\usepackage{textcomp}
\usepackage{gensymb}
\usetikzlibrary{patterns}

%%BOOKTAB
\setlength{\aboverulesep}{0pt}
\setlength{\belowrulesep}{0pt}
\setlength{\extrarowheight}{.75ex}
\setlength\parindent{0pt} %Rimuove indentazione


%%GEOMETRIA
\usepackage[a4paper]{geometry}
\newgeometry{inner=20mm,
            outer=49mm,% = marginparsep + marginparwidth 
                       %   + 5mm (between marginpar and page border)
            top=20mm,
            bottom=25mm,
            marginparsep=6mm,
            marginparwidth=30mm}
\makeatletter
\renewcommand{\@marginparreset}{%
  \reset@font\small
  \raggedright
  \slshape
  \@setminipage
}
\makeatother
 

%%COMANDI
\newcommand{\q}[1]{``#1''}
\newcommand{\lamb}[2]{\Lambda^{#1}_{\>{#2}}}
\newcommand{\norm}[1]{\left\lVert#1\right\rVert}
\newcommand{\hs}{\mathcal{H}}
\newcommand{\minus}{\scalebox{0.75}[1.0]{$-$}}
\newcommand{\hlc}[2]{%
  \colorbox{#1!50}{$\displaystyle#2$}}
\newcommand{\bb}[1]{\mathbb{#1}}
\newcommand{\op}[1]{\operatorname{#1}}
\renewcommand{\figurename}{Fig.}
\newcommand{\dom}[1]{D#1}
\newcommand{\avg}[1]{\left\langle{#1}\right\rangle}
\newcommand{\NN}{\mathbb N}
\newcommand{\RR}{\mathbb R}
\newcommand{\CC}{\mathbb C}
\newcommand{\mS}{\mathcal S}
\newcommand{\de}{d}
\newcommand{\abs}[1]{\left|#1\right|}
\newcommand{\deriv}[2]{\frac{\de #1}{\de #2}}
\newcommand{\pderiv}[2]{\frac{\partial #1}{\partial #2}}
\newcommand{\vect}[1]{\vec #1}
\newcommand{\kB}{k_B}
\newcommand{\sumn}{\sum_{i=1}^N}
\renewcommand{\ket}{\Ket}
\renewcommand{\bra}{\Bra}
\renewcommand{\braket}{\Braket}

\newcommand{\breakpage}{\begin{center}
  $\ast$~$\ast$~$\ast$
\end{center}}
\newcommand{\lesson}[2]{\marginpar{(Lezione #1 del #2)\onlyinsubfile{\\Compilata: \today}}}
\DeclareRobustCommand{\MQ}{{\small\textsc{MQ}}\xspace}
\DeclareRobustCommand{\MC}{{\small\textsc{MC}}\xspace}
%Prima era \small\textsc{MQ}\xspace

%%TESTATINE
\usepackage{fancyhdr}
\pagestyle{fancy}
\fancyhead{} % clear all header fields
\renewcommand{\headrulewidth}{0pt} % no line in header area
\fancyfoot{} % clear all footer fields
%\fancyfoot[R]{A.A. 2018/19} % other info in "inner" position of footer line
\cfoot{\thepage}


%%AMBIENTI
\theoremstyle{plain}
\newtheorem{thm}{Teorema}[section]
\newtheorem{lem}{Lemma}[section]
\newtheorem{prop}{Proposizione}[section]
\newtheorem{axi}{Assioma}
\newtheorem{pst}{Postulato}

\theoremstyle{definition}
\newtheorem{dfn}{Definizione}

\theoremstyle{remark}
\newtheorem{oss}{Osservazione}
\newtheorem{es}{Esempio}
\newtheorem{ex}{Esercizio}

%Spiegazioni/verifiche
\newenvironment{expl}{\begin{mdframed}[hidealllines=true,backgroundcolor=green!20,innerleftmargin=3pt,innerrightmargin=3pt,leftmargin=-3pt,rightmargin=-3pt]}{\end{mdframed}} %Box di colore verde

\newenvironment{appr}{\begin{mdframed}[hidealllines=true,backgroundcolor=blue!10,innerleftmargin=3pt,innerrightmargin=3pt,leftmargin=-3pt,rightmargin=-3pt]}{\end{mdframed}} %Approfondimenti matematici (box di colore blu)

%%Domande di Marchetti
\newtheorem{question}{Domanda}


%%OPERATORI
\DeclareMathOperator{\sech}{sech}
\DeclareMathOperator{\csch}{csch}
\DeclareMathOperator{\arcsec}{arcsec}
\DeclareMathOperator{\arccot}{arcCot}
\DeclareMathOperator{\arccsc}{arcCsc}
\DeclareMathOperator{\arccosh}{arcCosh}
\DeclareMathOperator{\arcsinh}{arcsinh}
\DeclareMathOperator{\arctanh}{arctanh}
\DeclareMathOperator{\arcsech}{arcsech}
\DeclareMathOperator{\arccsch}{arcCsch}
\DeclareMathOperator{\arccoth}{arcCoth} 

%%CONTATORI
\newcounter{Esercizio}
\stepcounter{Esercizio}

%%FANCY STUFF
\usepackage[Bjornstrup]{fncychap}
\usepackage[font=footnotesize, labelfont=bf,
            format=hang, labelformat=parens,
            labelsep=endash, justification=raggedright,
            singlelinecheck=on]{caption} 
\usepackage{subcaption} 
%\captionsetup[sub]{font=footnotesize,
%            labelsep=endash, justification=centering}
\usepackage{microtype}
\usepackage{lipsum}                     % Dummytext
\usepackage{xargs}                      % Use more than one optional parameter in a new commands
% 
\usepackage[colorinlistoftodos,prependcaption,textsize=tiny]{todonotes}
\newcommandx{\unsure}[2][1=]{\todo[linecolor=red,backgroundcolor=red!25,bordercolor=red,#1]{#2}}
\newcommandx{\change}[2][1=]{\todo[linecolor=blue,backgroundcolor=blue!25,bordercolor=blue,#1]{#2}}
\newcommandx{\info}[2][1=]{\todo[linecolor=OliveGreen,backgroundcolor=OliveGreen!25,bordercolor=OliveGreen,#1]{#2}}
\newcommandx{\improvement}[2][1=]{\todo[linecolor=Plum,backgroundcolor=Plum!25,bordercolor=Plum,#1]{#2}}
\newcommandx{\thiswillnotshow}[2][1=]{\todo[disable,#1]{#2}}
%
\usepackage[inline,marginclue,lang=italian]{fixme}


%%%%%%%%%%%%


\usepackage{imakeidx} %Indice analitico
\usepackage{hyperref} %hyperref va caricato sempre dopo footmisc, altrimenti le footnotes si buggano e riportano tutte alla prima pagina
\makeindex[columns=2, title=Indice analitico, options= -s indexstyle.ist]

\begin{document}
\renewcommand{\onlyinsubfile}[1]{}
\renewcommand{\notinsubfile}[1]{#1}

\newgeometry{total={170mm,257mm}, left=20mm, top=20mm}
%Impostazioni booktab (Rimuove quello spazio odioso tra righe)
\setlength{\aboverulesep}{0pt}
\setlength{\belowrulesep}{0pt}
\setlength{\extrarowheight}{.75ex}
\begin{center}
                \line (1,0){350} \\
                \textsc{\normalsize Trascrizione degli appunti delle lezioni di}\\
                [0.25in]
                \huge{\bfseries Introduzione alla teoria quantistica dell'informazione}\\
                [2mm]
                \textsc{\normalsize Tenute dal Prof. \textit{Simone Montangero}}
                \vspace{-0.5em}\\
                \textsc{\normalsize Presso l'Università di Padova}\\
                \vspace{-1em}
                \line (1,0){350} \\
        [0.2cm]
        \textsc{\normalsize A cura di: \textit{Francesco Manzali}, \textit{Mattia Morgavi}}\\
                \textsc{\normalsize Anno accademico 2018-2019}\\ 
        {\scriptsize Compilato il \today}
\end{center}

\newgeometry{inner=20mm,
            outer=49mm,% = marginparsep + marginparwidth 
                       %   + 5mm (between marginpar and page border)
            top=20mm,
            bottom=25mm,
            marginparsep=6mm,
            marginparwidth=30mm}

\makeatletter
\renewcommand{\@marginparreset}{%
  \reset@font\small
  \raggedright
  \slshape
  \@setminipage
}
\makeatother

\tableofcontents 
\clearpage
\chapter*{Introduzione}
Buonsalve!\\
In questo documento ho cercato di riordinare gli appunti del corso di Introduzione alla teoria quantistica dell'informazione tenuto dal professor Simone Montangero presso il Dipartimento di Fisica dell'Università di Padova nel corso del secondo semestre del 2018-19.\\
Potrebbero esserci errori di formattazione, parentesi saltate, o peggio, coefficienti/esponenti/segni errati in giro (ma non dovrebbero essere tanti). Se ne sgamate qualcuno, fatemi sapere. Ditemi anche (se avete tempo e non vi scoccia) se ci sono passaggi non chiari.\\
Prima di iniziare, ultimo disclaimer (che dovrebbe essere scontato dato che non ho una laurea): questi appunti non sono da intendere come sostituzione delle lezioni, o di altre dispense già presenti.\\
Buon viaggio! :)

\begin{flushright}
\textit{Francesco Manzali}, 20/02/2019
\end{flushright}
\clearpage
\section*{Aggiornamenti}
\begin{table}[hb]
    \centering
    \begin{tabular}{|cm{3cm}m{5cm}m{3cm}|}\toprule
        Data & Aggiunte & Errata corrige & Commenti\\\midrule
        \textbf{23/10/2018} & Prima pubblicazione & & \\
        \bottomrule
    \end{tabular}
    \caption{Cronologia di modifiche/aggiornamenti agli appunti}
    \label{updates}
\end{table}

\clearpage

\chapter{Circuiti quantistici}
\subfile{AppuntiGiornalieri/Febbraio/27_2.tex}

\subfile{AppuntiGiornalieri/Febbraio/28_2.tex}

\chapter{Effetti quantistici}
\subfile{AppuntiGiornalieri/Marzo/6_3.tex}

\subfile{AppuntiGiornalieri/Marzo/7_3.tex}

\subfile{AppuntiGiornalieri/Marzo/13_3.tex}

\subfile{AppuntiGiornalieri/Marzo/14_3.tex}

%Comandi utili
%\missingfigure{Testo}
%\fxnote{Test}, \fxwarning, \fxerror, \fxfatal (in ordine di gravità)

\listoffigures

\listoftables

%%BIBLIOGRAFIA
\clearpage
\begin{thebibliography}{9}
\bibitem{detection-free}
Kwiat et al.
\textit{Realistic Interaction-Free Detection of Objects in a Resonator}
Foundation of Physics, 1998

\bibitem{note-MQ}
Manzali Francesco, Mattia Morgavi
\textit{Trascrizione degli appunti del corso di \MQ tenuto dal prof. Pieralberto Marchetti}
\url{https://drive.google.com/open?id=1q0yrQNs-9_2EH3r6irQlpMr8MoiWNe-C}

\bibitem{landauer}
Eric Lutz, Sergio Ciliberto.
\textit{Information: From Maxwell’s demon to Landauer’s eraser}
Physics Today 68, 9, 30 (2015) \url{https://doi.org/10.1063/PT.3.2912}

\bibitem{zeno}
Saverio Pascazio
\textit{All you ever wanted to know about the quantum Zeno effect in 70 minutes}
2014, \url{https://arxiv.org/abs/1311.6645}

\bibitem{beam-splitter}
\url{http://www.pas.rochester.edu/~howell/mysite2/Tutorials/Beamsplitter2.pdf}
François Hénault
\textit{Quantum physics and the beam splitter mystery}
\url{https://arxiv.org/ftp/arxiv/papers/1509/1509.00393.pdf}

\bibitem{nand2tetris}
Noam Nisan, Shimon Schocken
\textit{From Nand To Tetris Course}
Coursera, \url{https://www.coursera.org/learn/build-a-computer/home/welcome}
\end{thebibliography}

%%INDICE ANALITICO
\printindex

\end{document}

